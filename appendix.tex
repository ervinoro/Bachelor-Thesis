\newpage
%\appendix
%\iflanguage{english}%
%  {\section{Appendix}
%  \addcontentsline{toc}{section}{Appendix}
%  }%
%  {\section{Lisad}
%  \addcontentsline{toc}{section}{Lisad}}

%\subsection{Glossary}
%\addcontentsline{toc}{subsection}{I. Glossary}
\begin{appendices}
\printglossaries

\newpage
%=== Licence in English
\newcommand\EngLicence{{%
\selectlanguage{english}
\section{Licence}

%\addcontentsline{toc}{subsection}{II. Licence}

\subsection*{Non-exclusive licence to reproduce thesis and make thesis public}

I, \textbf{\Author},

\begin{enumerate}
\item
herewith grant the University of Tartu a free permit (non-exclusive licence) to:
\begin{enumerate}
\item[1.1]
reproduce, for the purpose of preservation and making available to the public, including for addition to the DSpace digital archives until expiry of the term of validity of the copyright, and
\item[1.2]
make available to the public via the web environment of the University of Tartu, including via the DSpace digital archives until expiry of the term of validity of the copyright,
\end{enumerate}

of my thesis

\textbf{\topic}

supervised by \SupervisorName\ and \CosupervisorName.

\item
I am aware of the fact that the author retains these rights.
\item
I certify that granting the non-exclusive licence does not infringe the intellectual property rights or rights arising from the Personal Data Protection Act. 
\end{enumerate}


Tartu, \today
}}%\newcommand\EngLicence


%=== Licence in Estonian
\newcommand\EstLicence{{%
\selectlanguage{estonian}
\section*{II. Litsents}

\addcontentsline{toc}{subsection}{II. Litsents}

\subsection*{\topic}

Mina, \textbf{\Author},

\begin{enumerate}
\item
annan Tartu Ülikoolile tasuta loa (lihtlitsentsi) enda loodud teose

\textbf{\topic}

mille juhendajad on \JuhendajaName\ ja \KaasjuhendajaName

\begin{enumerate}
\item[1.1]
reprodutseerimiseks säilitamise ja üldsusele kättesaadavaks tegemise eesmärgil, sealhulgas digitaalarhiivi DSpace-is lisamise eesmärgil kuni autoriõiguse kehtivuse tähtaja lõppemiseni;
\item[1.2]
üldsusele kättesaadavaks tegemiseks Tartu Ülikooli veebikeskkonna kaudu, sealhulgas digitaalarhiivi DSpace´i kaudu kuni autoriõiguse kehtivuse tähtaja lõppemiseni.
\end{enumerate}


\item
olen teadlik, et punktis 1 nimetatud õigused jäävad alles ka autorile.
\item
kinnitan, et lihtlitsentsi andmisega ei rikuta teiste isikute intellektuaalomandi ega isikuandmete kaitse seadusest tulenevaid õigusi. 
\end{enumerate}


Tartus, \today
}}%\newcommand\EstLicence


%===Choose the licence in active language
\iflanguage{english}{\EngLicence}{\EstLicence}

\newpage
\section{Indirect function calls}
\label{apx:calls}
\begin{lstlisting}[style=asm]
// Experiment to quantify the exact memory/speed tradeoff when using 24 vs 32 bits per function pointer in function table
// Decision: 32 bits makes more sense in this case
.syntax unified
.cpu cortex-m7
.thumb

function: // Function to be called
bx lr // Immideately return

jump32: // How would function call overhead look like in case of 32 bits of RAM per function pointer
ldr r0, [r1]
bx r0

jump24: // How would function call overhead look like in case of 24 bits of RAM per function pointer
ldr r0, [r1]
lsr r0, r0, #8
orr r0, 0x08000000
bx r0

.type  Reset_Handler, %function
Reset_Handler:
ldr sp, =_estack
ldr r1, =function // Load function pointer into register 1
bl jump32
sub r1, #1 // Subtract one byte from the label address because 24 bits have to be right-aligned
bl jump24
loop:
b loop

.section  .isr_vector,"a" // Interrupt Service Routines vector table, "a"llocatable
g_pfnVectors:
.word _estack // End of stack; defined in linkerscript
.word Reset_Handler
\end{lstlisting}



\end{appendices}