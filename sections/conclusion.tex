\clearpage
\section{Conclusion}
\label{s:conclusion}

The aim of this thesis has been to design and implement a method of updating software more suitable for ESTCube-2 than existing alternatives. Based on literature review requirements for such system were identified and a design meeting those criteria proposed. The design was validated on prototype versions of ESTCube-2 hardware and software.

The designed system enables adding and updating single functions independently, therefore effectively minimizing the need to re-upload unchanged code. Additionally, it requires no on-board modifications to uploaded binaries. It does introduce performance overhead of three instructions per application function call, and estimated 2.25\% storage overhead. Some limitations were also discovered with the proposed design, most importantly the inability to function in the presence of large local variables.

Therefore, the overall result is that limitations present in other existing solutions and outlined in requirements were successfully mitigated. Proposed and implemented solution requires less uplink bandwidth than full image replacement, provides unlimited inter-component communication unlike virtualization solutions, and does not require any on-board modifications unlike binary differences or dynamic module loading. On the other hand, some existing solutions, like full image replacement for example, may be applicable in more different situations despite their shortcomings. Future work in this area should focus on overcoming shortcomings outlined in this thesis by exploring alternative compilers or exploring possibilities of adding additional code generation options to GNU C compiler.

Review of existing solution and in-depth analysis of one additional solution, presented in this thesis, can be used by ESTCube project and by other embedded systems to make better decisions about what methods to use for updating software. This can help those systems to better maintain their functionality in an uncertain environment, and to even take advantage of uncertainty by adding novel applications or improving functionality during their missions.
