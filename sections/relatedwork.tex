\newpage
\section{Related work}
\label{s:relatedwork}

The problem
of software updates on embedded systems has seen many different solutions so far. A brief overview of existing methods is given in this section.

\subsection{Full system image replacement}

% generic overview of what the method is (+ ESTCube-1)
For example, on ESTCube-1, one way to update on-board code was to recompile the entire code-base, upload it to the satellite, and after rebooting the \gls{mcu} this new code would be active \cite{Suenter2016}. This method benefits from a simple design. Additionally, by replacing the entire system image, any compatibility issues are eliminated and next to none on-board processing is required. Support for full image replacement is often implemented as a fallback even on systems that support other more advanced update mechanisms as well \cite{Tarbe2013,Greco2005,Garrido1998}.

% dual images
Additionally, several missions have had the ability to keep several such system images stored at any time, allowing switching between them in case of problems. In some cases, like MINISAT01, one of the images was read-only and only the other one could be modified \cite{Garrido1998}. However, in some other cases, like  ESTCube-1 \cite{Tarbe2013} and The Mars Exploration Rovers \cite{Greco2005}, it has been possible to update both firmware images independently. The former poses significant drawbacks: in the case of MINISAT01, a complete firmware update took 2 full days, and for that duration the original launch firmware version had to be used \cite{Garrido1998}.

% drawbacks + ESTCube-1
However, this kind of simple design also poses significant drawbacks. Most importantly, it requires large amounts of uplink, even when the change was minor. For example, on ESTCube-1, taking into account the size of the firmware, uplink speed and orbital parameters, uninterrupted firmware update would take about 1.5 days to complete \cite{Suenter2014}, even though on average an update changed only about 2\% of the code. In the case of the MINISAT01, a full firmware update took 2 complete days, while a partial update could be done in few hours \cite{Garrido1998}. In the case of the Mars Exploration Rovers, replacement of the entire firmware image required uplinking about 8 megabytes of data, while a delta update that they completed required uplinking of approximately only 2 megabytes \cite{Greco2005}.

\subsection{Virtual machines and script interpreters}

Another method that is commonly used when new code needs to be uploaded frequently and in small parts, is to utilize virtualization techniques on board. For example, interpreted scripting language pawn script was used on ESTCube-1 \cite{Suenter2016} and will be used on TTÜ100 satellite \cite{Aasavaeli2017}. With this approach some parts of the code are written in native code and some as scripts. Since scripts interact with rest of the system only through predefined \glspl{api}, they can be modified without influencing any other parts of the system. Thanks to the abstraction layer that the script interpreter provides, script files can be moved and rearranged without difficulties \cite{Riemersma2017}.

As an alternative to scripting languages, virtual machines have also been used to achieve similar results. For example, \textcite{Simon2006} describe a Java virtual machine for wireless sensor networks. Custom bytecodes specifically tailored for embedded systems have been explored as well \cite{Levis2002}.

However, not all of the on-board software can be rewritten this way. Interpreted code can be significantly slower than compiled code. For example, \textcite{Simon2006} measured compiled C code to be 10 times faster that equivalent java byte-code running in their interpreter (however, this benchmark can not be considered conclusive). Additionally, virtualized software has limited access to other system resources through predefined \glspl{api}.

\subsection{Binary differences}

In order to overcome the large uplink requirements of full image replacement while keeping the benefits of native code (most notably maximum execution speed and unlimited inter-component communication), some authors have experimented with calculating a delta between two system images and using it to recreate the new firmware image on board.

One of the simplest binary difference based approaches for delta software updates was implemented on MINISAT01. Their entire firmware image consisted of 32 parts, each of which could be updated separately. While updating the whole firmware took two full days, updating a single part could be done in just few hours. However, this approach had a serious limitation: old and updated code had to be exactly binary compatible, no lengths could be changed. Therefore this kind of approach is only useful for updating values of constants, since code updates typically also change the length of compiled binary. \cite{Garrido1998}

To overcome this limitation, MINISAT01 also featured third method of updating on-board software. With this method, new or updated functions were located at the end of the firmware, while old versions were kept on their original locations. Then any calls to modified functions were changed in on-board software. This was possible, since modifying function call with new location does not change the length of the call, and MINISAT01 stored its firmware in \gls{eeprom}. \cite{Garrido1998}

A more complex approach to binary difference based software updates was taken on the Mars Exploration Rovers. They calculated differences between new and old firmware images that could start and end at arbitrary locations and change the size of the image as well. However, to overcome the limitations of flash memory (described in Section~\ref{s:hardware}), they had to build the updated image in volatile memory and then store the new generated image in non-volatile memory. This meant that they had to reserve a full day to assemble, validate and store the updated software. During that time no other activities could be carried out. They also acknowledge that ``The single biggest improvement to the Mars Explorer Rovers' flight software modification process would be to reduce the amount of time necessary to stand down from nominal surface operations.'' \cite{Greco2005}

\subsection{Dynamically loaded modules}

Already for some time general-purpose computer systems have supported loading software components independently of each other by utilizing virtual memory \cite{Kilburn1962}. This way each component can be statically linked beforehand without conflicts. Embedded systems that have a \gls{mmu}, have similarly used server-side pre-linking of software \cite{Shen2010}.

A very similar problem to that of embedded systems without an \gls{mmu} has been solved on general-purpose computers for shared libraries. While on embedded systems modifications to compiled binaries are undesirable due to the limited computational power and the use of flash memory, on general-purpose computers modifications to binaries would force several versions of the same library to be kept in memory. Neither can be statically linked in order to avoid conflicts between different modules. The solution is to compile software in a position independent manner by introducing a level of indirection between symbols and actual memory addresses. Several different approaches to such indirection have been explored \cite[Chapter~8]{Levine1999}. Hybrid solutions of pre-linking and pre-locating combined with dynamic re-location have also been explored \cite{Dong2009}.

The table that maps symbols to addresses can be part of the \gls{os} memory or part of each module. Example of the former can be found in the SOS operating system described by \textcite{Han2005}, where each module has to register functions into a jump table. The latter approach has been implemented, for example, in Contiki OS, that makes use of modified \gls{elf} to hold code alongside its \gls{got} \cite{Dunkels2006}. Unfortunately, dynamic linking of \gls{elf} files is still not widely supported on embedded operating systems \cite{Xinyu2017}. Among others like it, the standard distribution of FreeRTOS (the operating system used on ESTCube-2 \gls{obc}) does not support loading modules dynamically \cite{Barry2005}. \textcite{Xinyu2017} describe a method for including the code relocation functionality into the relocatable code itself as one possible solution.
