\newpage
\section{Other work}

\todo{copied from abstract}Full system image replacement, like what was used on ESTCube-1 \cite{Tarbe2013,Suenter2016}, consumes large amounts of bandwidth and is, therefore, not viable very frequently. Virtual machine or script interpretator based systems have also been implemented for satellites like on TTÜ100 \cite{Aasavaeli2017} and ESTCube-1 \cite{Ehrpais2016}, but these lack the performance of native code and are, therefore, not usable for performance critical functionality. Approaches based on binary differences, like the one that was used on Mars exploration rovers \cite{Greco2005}, combine low bandwidth requirements with native performance, but require rewriting of entire firmware image, which is problematic, since ESTCube-2 on-board computer stores firmware on flash memory with large erase blocks and limited number of write cycles \cite{Haljaste2017}. To overcome these shortcomings, different systems, like StrandMind \cite{Bridges2013}, SenSpire OS \cite{Dong2009}, and Contiki \cite{Dunkels2006}, have implemented different schemes for dynamically loading modules, but the operating system for ESTCube-2 on board computer, FreeRTOS, does not support dynamically loaded software components out of the box \cite{Barry2005}. \todo{this list is not exchaustive}

