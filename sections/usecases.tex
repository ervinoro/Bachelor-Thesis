\newpage
\section{Use cases}
\label{s:usecases}

Use cases for the ESTCube-2 \gls{obc} software updating system (predictions on how the system will need to be used later on) are derived from ESTCube-1 \gls{cdhs} experience. Differences between the two missions are taken into account where applicable.

\subsection{ESTCube-1 experience}

ESTCube-1 \gls{cdhs} received 20 distinct firmware updates during its mission, with 19 of them being successful \cite{Suenter2016}. \textcite{Slavinskis2015} list among functionalities added to the \gls{cdhs} ``power saving mode, variety of data logging functions, high time-resolution functions for sensor measurements, experiment-related functions, additional preprocessing of attitude measurements, as well as attitude determination and control algorithms.'' Some of the changes were introduced in response to unforeseen circumstances.

Size of the changes in corresponding source codes was analyzed. On average, 2.09\% ($\sigma$=3.43\%) of the code lines were added, removed or edited during a firmware update. The largest update changed 14.71\% of the code, while the smallest update only modified 0.02\% of all the lines. Updates with largest code changes were so due to addition of large portions of new code. Overall, size of the source code for \gls{cdhs} software (including all libraries) also gradually increased through the mission (from 63~390 lines of code to 78~868, a 24.42\% increase). However, no update consisted of just new code being added, always some previously existing code was changed or removed as well.

As it was on ESTCube-1 (a classic nonsegmented firmware), a minor code modification could result in a binary difference of almost 99\%, which caused the need to upload an entire firmware image again every time. Size of a typical \gls{cdhs} firmware image after compilation was 250 kibibytes. Those images had a Shannon entropy of about 0.6 (on the scale of 0$-$1), resulting in a theoretical maximum compression ratio of 40\%. \cite{Suenter2016}

\subsection{Use cases during ESTCube-2 mission}

\subsubsection{Hardware, on which the software will be updated}\label{s:hardware}

ESTCube-2 \gls{obc} will be centered around an STM32F767IIT6 \gls{mcu}, which has 2 mebibytes\footnotemark of internal flash and 516 kibibytes\footnotemark[\value{footnote}] of internal \gls{sram}. For the data of running programs the \gls{obc} has 2 mebibytes\footnotemark[\value{footnote}] of \gls{mram}. For external configuration tables, error logs, on-board statistics, and other data without strict latency restrictions, it has 3$\times$512 kibibytes\footnotemark[\value{footnote}] of \gls{fram}. Mass storage for firmware versions, measurements, and payload data is provided by 2$\times$32 mebibytes\footnotemark[\value{footnote}] of external flash.
\cite{Haljaste2017}

At least critical software components must be stored in the internal flash, so that it would be possible to disable all external device drivers in safe mode. This is desirable since having more code enabled increases the probability of any faults occurring. However, flash memory consists of sectors, which can be up to 256 kibibytes\footnotemark[\value{footnote}] in size \cite{STMicroelectronics2018}. In order to edit any data already written to the memory, an entire sector must be erased and rewritten \cite{STMicroelectronics2018}.

\footnotetext{\label{fn:units}Cited sources claim those numbers to be in decimal units (kilo- and megabytes), but are assumed to do so by mistake.}

\subsubsection{Types of updates}

Due to the limited number of suitable launches (caused mostly by limited funding possibilities), it might happen that the satellite has to be delivered on an unexpectedly accelerated schedule, so that some software functionalities are not completed or sufficiently tested before the launch. In such case, significant amounts of new code would need to be added to the firmware with an update. Additionally, some previous code would need to be modified to make use of those new functionalities. However, largest parts of the firmware by size - \gls{os}, \gls{hal} and drivers - must definitely be finalized before the launch in order to enable successful satellite operation. This means that the size of even the largest update caused by this reason would stay under about 20\%.

Several software components are experimental. In order to assess the properties of novel solutions, they need to be compared with existing methods, which could mean the need to deploy alternative algorithms for some period of time. However, swapping out a software component could not cause changes larger than introducing a new feature.

The testing of novel software solutions also entails the need for iterative improvements, as the perfect setup is unlikely to be achieved on the first try. While most of this should be possible by only changing configuration values separate from the firmware, it might happen that some unforeseen change does require code rewriting. Updates for aforementioned reasons would affect only a small number of components and would alter significantly less than 1\% of the firmware.

When bugs are discovered in on-board software, a fix needs to be developed and deployed. Such update would mostly consist of changes to the existing code, and can be expected to change considerably less than 1\% of the firmware. Changes would also be limited to a single component or functionality. When a bug is discovered, it may be necessary to deploy the fix without delays, to minimize the probability of any mission critical risks materializing.

Unforeseen issues are also possible with the satellite hardware, due to the use of \gls{cots} components, extreme miniaturization, and the educational nature of the development process. Some of such issues could be circumvented with software workarounds. These can range from simple code changes to the addition of completely new functionality.

Lastly, the main mission of ESTCube-2 is to conduct experiments, like unreeling the plasma brake tether. During experiments it might be useful to rapidly deploy new small subroutines that handle aspects of the experiment that were not predicted beforehand.
