\newpage
\section{Testing}
\label{s:testing}

A build methodology was used to explore the feasibility of proposed system. After implementation, viability of the solution was tested by simulating updates to a development version of the firmware running on a prototype board.

\subsection{Implementation details}

A set of Python scripts aids with function compilation and packaging. Functions are discovered by searching function definitions from all source files. Global variables are extracted from symbol tables after all sources have been compiled. They are handled differently because function arguments and their types can not be obtained from symbol tables, but since all information about global variables is available there, it is preferable to let compiler do the source parsing. A side effect of implementing function discovery with regular expressions is the inclusion of function definitions from within block comments. This, however, should not cause any issues, since they will not have a corresponding binary section and therefore they will not be included in any snapshots. Data about functions and global variables, along with respective object files, gets stored in an SQLite3 database. Scripts also exist for generating interception macros, linker script, and packages with headers.

Function table can not be implemented as a branch table, since internal flash and \gls{sram} in the \gls{mcu} are too far away in the memory space. Function table will therefore contain pointers, 4~bytes per function. Storing 3~bytes per function would require 2~instructions of additional overhead per function call, so it has been decided that 4~bytes per function will be stored (Appendix~\ref{apx:calls}). Pointers will point to the first instruction of the function (right after the header).

Firmware contains a table generation function (Appendix~\ref{apx:gentable}) that must run before any application software gets called. It walks over package headers through the space allocated for such packages, jumping over function bodies based on length, and stops on the first header with the type 'nothing' (\texttt{0xff}, Table~\ref{tab:header}). Function pointers are added to the function table, and global variables get copied to their respective allocated memory areas.

\subsection{Function discovery and preparation}

Aforementioned Python scripts were tested against simple test functions, as well as functions from the \gls{aocs} repositories. For discovering global variables, two methods were compared: parsing source code variable definitions, and compiling source with \texttt{-fdata-sections} to discover global variable names from symbol table. Former method was not sufficiently effective due to large amount of different global variables, including matrices and other complex types. Using the latter method, most functions and all global variables were successfully packaged.

\subsection{Running application functions}

Since linking was known to be a difficult aspect, the proposed solution was first tested in a situation with minimal number of external symbols. Once this was observed to be functional, a more typical use case was analyzed as well.

\subsubsection{Fully self-contained function}

As the first test, a fully self contained function was written, meaning one that did not depend on any external variables or functions, including any standard library or operating system methods. This function blinked an indicator light by writing appropriate values directly to the \gls{mcu} registers.

It was compiled independently of the rest of the firmware and stored in flash memory following the firmware image. Firmware successfully discovered the package, populated the table of function offsets, and called the function through this table. A minor inconvenience was noted: debug symbols for applications functions are not available to the debugger from the firmware's \gls{elf}.

\subsubsection{Calling firmware functions}

Next, a function was written that blinks the indicator lights by calling appropriate methods from \gls{hal} \gls{api}. All arising issues were mitigated and the test was concluded mostly successfully. Following difficulties were experienced:

Firstly, author was unable to find a way to tell the GNU linker \texttt{ld} to use symbol addresses from a map file generated by prior linking. Even though \textcite{Dunkels2006} claim to have used just the map file for pre-linking, no information on how to achieve that was found by the time of writing from documentation, different online sources nor an online forum (\url{https://stackoverflow.com/q/48028126/7088748}). This resulted in the need to link firmware separately once, and then again with each application function, hoping that the firmware layout stays unchanged. Initially that was not the case, but after explicitly specifying section order in the linker script, layout consistency was achieved.

Secondly, issues arose with the way the GNU C compiler handles local constants. \Gls{hal} functions for accessing \gls{gpio} pins take a struct of port letter and pin number as an argument. Unfortunately, compiler placed the structs to \texttt{.rodata} section separately from the function code, even though they were local. More about that issue can be found in Section~\ref{s:rodata}. For the purpose of this test, the issue was mitigated by storing all port letters and pin numbers in single-byte local variables, which the compiler then decided to store directly within the code section.

\subsection{Performance}

Performance overhead with the implemented system is three additional instructions per function call (Appendix~\ref{apx:calls}). Storage overhead per function is 12 bytes (Table~\ref{tab:header}).

An in-development \gls{aocs} library was packaged, and results were analyzed. Average size of a compiled function was 533 bytes ($\sigma$=1194 bytes) (Figure~\ref{fig:kalman}). Largest function was just over 8.5 kilobytes. On the other hand, the library also contained several extremely small functions. For example, one function (shown in Appendix~\ref{apx:cos}) was just 28 bytes after compilation, therefore the 12 byte header caused a 30\% storage overhead. However, this function could be inlined, replaced with a preprocessor macro, or replaced with longer a implementation in later versions of this \gls{aocs} library.

\begin{figure} [htb]
\begin{tikzpicture}
\begin{axis}[
    ymin=0, ymax=18,
    xmin=0, xmax=8600,
    axis y line=left,
    axis x line*=bottom,
    area style,
    width=0.96\textwidth,
    height=0.4\textwidth,
    xlabel = {size of binary (bytes)},
    ylabel = {count},
    first y axis line style=blue,
    ylabel shift = -6 pt
    ]
    \pgfplotsset{compat = 1.3}
    \pgfplotsset{every x tick label/.append style={font=\small, yshift=0.3ex}}
    \pgfplotsset{every y tick label/.append style={font=\small, xshift=0.3ex}}
    \addplot+[ybar interval,mark=no] plot coordinates { (0, 1) (50, 13) (100, 17) (150, 9) (200, 4) (250, 4) (300, 5) (350, 4) (400, 1) (450, 2) (500, 2) (550, 0) (600, 1) (650, 6) (700, 1) (750, 2) (800, 2) (850, 2) (900, 1) (950, 0) (1850, 1) (1900, 0) (2650, 1) (2700, 0) (6550, 1) (6600, 0) (8500, 1) (8550, 0) };
\end{axis}

\begin{axis}[
    ymin=0, ymax=18,
    xmin=0, xmax=8600,
    area style,
    width=0.96\textwidth,
    height=0.4\textwidth,
    hide x axis,
    axis y line=right,
    ylabel = {storage overhead (\%)},
    second y axis line style={red},
    ylabel style={yshift = 8 pt}
    ]
    \pgfplotsset{compat = 1.3}
    \pgfplotsset{every x tick label/.append style={font=\small, yshift=0.3ex}}
    \pgfplotsset{every y tick label/.append style={font=\small, xshift=-0.3ex}}
    \addplot[red,domain=0:8600,samples=430]{1200/(12+x)};
\end{axis}


\end{tikzpicture}
\caption{Compiled size and respective storage overhead for functions from an in-development \gls{aocs} library.}
\label{fig:kalman}
\end{figure}

The overall storage overhead caused to the tested software version by function headers was 2.25\%. Since very small functions could be eliminated at later stages of development, the actual storage overhead can be assumed to be even lower.

\subsection{Section .rodata}\label{s:rodata}

ESTCube is using GNU Tools for Arm Embedded Processors (version 7.2.1 at the time of writing) to compile \gls{obc} software. When the GNU C compiler finds local constants that are longer than a byte, it separates them from the code and creates a \texttt{.rodata} section for them. The \texttt{.text} section will then contain relative reference (\texttt{//1}) to an absolute address (\texttt{//2}) inside the \texttt{.text}, which in turn points to the data (\texttt{//3}) inside \texttt{.rodata} (Figure~\ref{fig:rodata}). This structure remains even when the linker is told to put the \texttt{.text.function} and the corresponding \texttt{.rodata} into a single output section.

\begin{figure} [ht]
\begin{lstlisting}[language=C]
int function() { int a[] = {97, 98, 99, 100, 101, 103}; }
\end{lstlisting}
\begin{lstlisting}[style=asm]
// Undesirable:
0000 <.text.function>:
/---/
6:  4b07    ldr     r3, [pc, #28] ; (24 <function+0x24>) //1
/---/
24: 0000    .word   0x0000 //2 absolute address of .rodata,
                           //  currently not linked
0000 <.rodata>: // section currently not yet relocated
0:  0061    .word   0x0061 //3
4:  0062    .word   0x0062
8:  0063    .word   0x0063
c:  0064    .word   0x0064
10: 0065    .word   0x0065
14: 0067    .word   0x0067
\end{lstlisting}
\caption{Local constants after compilation with GNU C compiler}
\label{fig:rodata}
\end{figure}

Even when using \gls{pic}, the code uses expressions relative to the \gls{pc} to access the \gls{got}, but the \gls{got} still contains absolute memory addresses to \texttt{.rodata}. However, in all cases, absolute memory addresses to application function components, even when in the \gls{got}, are undesirable, because by requirements, code should not require on-board modifications and should be movable in memory space.

Research on documentation, an online forum (\url{https://stackoverflow.com/q/45371949/7088748}) and the compiler support mailing list (\url{https://gcc.gnu.org/ml/gcc-help/2018-03/msg00015.html}) have not revealed any potential solutions by the time of writing.

This is not a fundamental problem with the proposed approach, since the compiler is already storing data (absolute memory addresses in this case) at the end of the \texttt{.text} section, and referencing it by \gls{pc}-relative expressions. The local constants could themselves be stored in the same way. Instead, this problem is caused by the fact that nobody has happened to implement a compiler flag to turn of the generation of \texttt{.rodata} section for local constants. This is probably due to lack of use cases prior to this work.
