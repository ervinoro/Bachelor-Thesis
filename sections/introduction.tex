\newpage
\section{Introduction}

\TODO{Write introduction}
\TODO{Mention, that the partial software updating system is being developed for the OBC, even though it could theoretically be used on other systems as well.}
% What is it in simple terms (title)?
% Why should anyone care?
% What was my contribution?
% Full system image replacement, like what was used on ESTCube-1 \cite{Tarbe2013,Suenter2016}, consumes large amounts of bandwidth and is, therefore, not viable very frequently. Virtual machine or script interpretator based systems have also been implemented for satellites like on TTÜ100 \cite{Aasavaeli2017} and ESTCube-1 \cite{Ehrpais2016}, but these lack the performance of native code and are, therefore, not usable for performance critical functionality. Approaches based on binary differences, like the one that was used on Mars exploration rovers \cite{Greco2005}, combine low bandwidth requirements with native performance, but require rewriting of entire firmware image, which is problematic, since ESTCube-2 on-board computer stores firmware on flash memory with large erase blocks and limited number of write cycles \cite{Haljaste2017}. To overcome these shortcomings, different systems, like StrandMind \cite{Bridges2013}, SenSpire OS \cite{Dong2009}, and Contiki \cite{Dunkels2006}, have implemented different schemes for dynamically loading modules, but the operating system for ESTCube-2 on board computer, FreeRTOS, does not support dynamically loaded software components out of the box \cite{Barry2005}. \todo{this list is not exchaustive}

%Perhaps a list of thesis objectives here - what is presented in this thesis. The list that the defence committee would keep track of as a checklist, to see if the thesis actually fulfils all of the promised objectives.

%Then could describe the organisation - which objective corresponds to which section, etc.

The rest of this paper is organized as follows.
\TODO{What you are doing in each section (a sentence or two per section)}


\subsection{ESTCube-2}

ESTCube-2 is an experimental three-unit CubeSat\todo{glossary?}. Its main missions are testing tether module for blasma brake deorbiting \todo{footnotes?}(previous versions of which have flown on the satellites ESTCube-1 and Aalto-1), Earth observation camera system (which is based on the \gls{eseo} camera), high speed C-band downlink system, and a novel miniaturized (up to 0.6 units) satellite bus. Other payloads include cold gas propulsion module by NanoSpace, and thin film protective coating experiment by Captain Corrosion OÜ. The ESTCube-2 mission is also planned to serve as an in-orbit demonstration of the platform, which could then be employed on future missions. \cite{Iakubivskyi2016}

ESTCube-2 is developed mostly by the students of University of Tartu and students that join the ESTCube program from all over the world \cite{Ehrpais2016}. Also among the main objectives for ESTCube-2 is to educate a new generation of space engineers, and to promote space technologies in general \cite{Iakubivskyi2016}.

The \gls{obc} of the satellite is tasked with running \gls{aocs} algorithms, controlling payload experiments and startracker, and handling telemetry and telecommands. The most important requirements for \gls{aocs} are set by blasma brake and Earth observation payloads. The former needs sufficient enough angular momentum for the centrifugal deployment of the tether, and the latter requires accurate pointing. While some methods, like the use of magnetic torquers, build upon the heritage of the successful ESTCube-1 mission, others, like reaction wheels by Hyperion Technologies and in-house developed startracker, are completely new for the team. Due to the large amount of experimental software, it is planned that it should be possible to perform firmware updates on all of the \glspl{mcu} after the launch of the satellite. This enables the team to correct any unexpected problems that the satellite may encounter. \cite{Ehrpais2016}
