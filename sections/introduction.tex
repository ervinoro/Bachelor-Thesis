\newpage
\section{Introduction}

Main goal of this thesis is to design and implement a method of updating software more suitable for ESTCube-2 than existing alternatives. Proposed solution focuses on enabling delta updates (avoiding upload of unchanged code as much as possible) and reducing the amount of on-board processing required. Additionally, updatable software's performance nor abilities should be limited either.

During the ESTCube-2 mission, loading of new software onto the satellite is planned, in order to introduce new features, test and compare novel software solutions, and resolve potential software or hardware issues. The value of increased software flexibility is measurable and can be higher than, for example, that of hardware flexibility \cite{Nilchiani2009}. For updating software on ESTCube-2, main difficulties are slow uplink (9600 bits per second), limited on-board processing power \cite{Ehrpais2016}, execution of software from Flash memory \cite{Haljaste2017} (limitations of which are detailed in Section~\ref{s:hardware}), and frequent updates due to the missions experimental nature.

All existing solutions for updating software have limitations (described in Section~\ref{s:relatedwork}) when applied to the \gls{obc} of ESTCube-2. The topic is relevant right now, since ESTCube-2 is currently in development phase. Even though this thesis focuses on the use cases of ESTCube-2, the topic has wider importance - several existing solutions have been designed with Internet of Things or wireless sensor networks \cite{Dunkels2006,Han2005} in mind. Those systems have several common aspects with nanosatellites. The method proposed in this thesis has not been previously described in literature.

Scope of this work is limited to on-board software written in C, GNU toolchain, and standard FreeRTOS distribution.

\subsection{ESTCube-2}

ESTCube-2 is an experimental three-unit CubeSat. The mission is planned to serve as an in-orbit demonstration of the platform, which could then be employed on future missions. \cite{Iakubivskyi2016}

\textcite{Iakubivskyi2016} list among the systems to be tested on ESTCube-2
\begin{itemize}
  \item tether module for plasma brake deorbiting (previous versions of which have flown on the satellites ESTCube-1 and Aalto-1)
  \item Earth observation camera system (which is based on the \gls{eseo} camera)
  \item high speed C-band downlink system
  \item a novel miniaturized (up to 0.6 units) satellite bus
  \item cold gas propulsion module by NanoSpace,
  \item thin film protective coating experiment by Captain Corrosion OÜ.
\end{itemize}


ESTCube-2 is developed mostly by the students of University of Tartu and students that join the ESTCube program from all over the world \cite{Ehrpais2016}. Also among the main objectives for ESTCube-2 is to educate a new generation of space engineers, and to promote space technologies in general \cite{Iakubivskyi2016}.

The \gls{obc} of the satellite is tasked with running \gls{aocs} algorithms, controlling payload experiments and star tracker, and handling telemetry and telecommands. The most important requirements for \gls{aocs} are set by plasma brake and Earth observation payloads. The former needs sufficient enough angular momentum for the centrifugal deployment of the tether, and the latter requires accurate pointing. While some methods, like the use of magnetic torquers, build upon the heritage of the successful ESTCube-1 mission, others, like reaction wheels by Hyperion Technologies and in-house developed star tracker, are completely new for the team. Due to the large amount of experimental software, it is planned that it should be possible to perform firmware updates on all of the \glspl{mcu} after the launch of the satellite. This enables the team to correct any unexpected problems that the satellite may encounter. \cite{Ehrpais2016}


\subsection{Outcomes}

Primary difficulty tackled in this thesis is how to add new or updated functions incrementally without the need for on-board modifications. Taking into account that and the scope of this thesis, main subject of research is how GNU C compiler generates machine code, and what are the options to alter the process to suit specific needs. Modifying the compiler itself is outside of the scope of this thesis, but in addition to compiler configuration flags, pre- and post-processing are utilized.

The method proposed in this thesis considers a function to be an atomic unit. Functions would be compiled separately and only new or updated functions would be uploaded.  This way less data needs to be uploaded while preserving software's native performance and abilities. Received functions would be stored sequentially in Flash memory without modifications. The need to delete Flash memory for each update is eliminated, making it possible to add new functions even without rebooting the embedded system. Performance, applicability and limitations of proposed solution are analyzed.

The rest of this thesis is organized as follows. Section~\ref{s:usecases} lists use cases based on ESTCube-1 experience and planned ESTCube-2 hardware. Section~\ref{s:relatedwork} describes identified four categories of previous approaches to the problem of embedded software updates. Strengths and weaknesses of them are listed. Section~\ref{s:design} describes the solution proposed in this thesis along with detailed requirements. Section~\ref{s:testing} includes results of testing the solution. Appendix~\ref{acronym} lists all used abbreviations. Appendix~\ref{apx:calls} has an experiment that explains, why function table was implemented with four bytes per function. Appendix~\ref{apx:gentable} includes the bootloading procedure needed to make proposed solution work. Appendix~\ref{apx:cos} demonstrates an example of a very short function from an in-development \gls{aocs} library.
